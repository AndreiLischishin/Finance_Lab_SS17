\section*{}

\begin{frame}[c]
	\frametitle{Ausblick I - Weiterführende Forschungsthemen}
	\begin{itemize}
		\item Verbesserung des Verfahrens\footnote{\glqq A new approach to unbiase estimation for SDE’s\grqq\ von Rhee und Glynn (2012)}:
		\newline
		Durch welche Wahl von $Y_l$ wird die Varianz noch geringer?
		\newline
		\item Nicht geometrische Levelsequenz\footnote{\glqq A model and variance reduction method for computing statistical outputs of stochastic elliptic partial differential equations\grqq\ von Giles et al. (2014)}:
		\newline
		Wie können die Levels und die Anzahl der Samples pro Level nun ausgewählt werden?
		\newline
		\item Multilevel Richardson-Romberg Extrapolation\footnote{\glqq Multi-level Richardson-Romberg extrapolation\grqq\ von Lemaire und Pag\`{e}s}:
		\newline
		In welchem Fall bringt dieses Verfahren eine weitere Verbesserung der MLMC-Methode?
	\end{itemize}
\end{frame}

\begin{frame}[c]
	\frametitle{Ausblick II - Weiterführende Forschungsthemen}
	\begin{itemize}
		\item Mehrdimensionale Zufallsvariable $P$\footnote{Beispielsweise zu finden in \glqq Complexity of Banach Space Valued and Parametric Integration\grqq\ von Daun und Heinrich (2013)}:
		\newline
		Wie überträgt sich das MLMC-Theorem auf diesen Fall?
		\newline
		\item Multilevel Quasi Monte Carlo\footnote{\glqq Multilevel quasi-Monte Carlo path simulation\grqq\ von Giles und Waterhouse (2009)}:
		\newline
		Wie verhält sich hierbei der Rechenaufwand?
		\newline
		\newline
		\item \textbf{Kommende Vorträge:} Wie sieht die MLMC-Methode in konkreten Anwendungen aus?
	\end{itemize}
\end{frame}